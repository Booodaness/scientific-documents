\documentclass[9pt,handout]{beamer}

% beamerthemeFeng.sty
% style file for beamer presentation

% tikz is used to ``draw'' title page and other templates in beamer
\usepackage{tikz,etoolbox}
\usetikzlibrary{shapes,arrows}

\definecolor{UWBlack}{HTML}{000000}
\definecolor{UWWhite}{HTML}{FFFFFF}


\definecolor{UWMathPinkL1}{HTML}{FFBEEF}
\definecolor{UWMathPinkL2}{HTML}{FF63AA}
\definecolor{UWMathPinkL3}{HTML}{DF2498}
\definecolor{UWMathPinkL4}{HTML}{C60078}
\definecolor{UWGrayL1}{HTML}{DFDFDF}
\definecolor{UWGrayL2}{HTML}{A2A2A2}
\definecolor{UWGrayL3}{HTML}{787878}
\definecolor{UWGrayL4}{HTML}{000000}
\definecolor{UWGoldL1}{HTML}{FFFFAA}
\definecolor{UWGoldL2}{HTML}{FFEA3D}
\definecolor{UWGoldL3}{HTML}{FFD54F}
\definecolor{UWGoldL4}{HTML}{E4B429}

\definecolor{carrot}{HTML}{EE693F}
\definecolor{ivory}{HTML}{F1F3CE}
\definecolor{emerald}{HTML}{265C00}
\definecolor{turquise}{HTML}{5BC8AC}
\definecolor{peacockblue}{HTML}{1E656D}
\definecolor{spicy}{HTML}{B51D0A}
\definecolor{bluegreen}{HTML}{5F968E}
\definecolor{rust}{HTML}{9B4F0F}
\definecolor{burntorange}{HTML}{DE7A22}
\definecolor{sea}{HTML}{20948B}
\definecolor{lagoon}{HTML}{6AB187}


% Set colors for different components in a slide
\setbeamercolor{background canvas}{bg=UWWhite}
\setbeamercolor{author}{fg=UWGoldL3}
\setbeamercolor{institute}{fg=UWMathPinkL3}
\setbeamercolor{title}{fg=UWGrayL4}
\setbeamercolor{section in head/foot}{bg=UWBlack, fg=UWGoldL3}
\setbeamercolor{author in head/foot}{fg=UWGoldL3, bg=UWBlack}
\setbeamercolor{title in head/foot}{fg=UWBlack,bg=UWGoldL3}
\setbeamercolor{institute in head/foot}{fg=UWGoldL3, bg=UWBlack}
\setbeamercolor{navigation symbols}{fg=UWBlack}
\setbeamercolor{normal text}{fg=UWGrayL3}
\setbeamercolor{section in toc}{fg=emerald}
\setbeamercolor{subsection in toc}{fg=bluegreen}
\setbeamercolor{frametitle}{fg=UWMathPinkL2, bg=UWGrayL1}
\setbeamercolor{block title}{bg=emerald, fg=ivory}
\setbeamercolor{block body}{bg=peacockblue!20, fg=peacockblue}
\setbeamercolor{section number projected}{bg=turquise,fg=black}
\setbeamercolor{block title example}{fg=rust,
	bg= sea!40}
\setbeamercolor{block body example}{fg= burntorange,
	bg= lagoon!20}

\setbeamerfont{frametitle}{series=\bfseries} % bold frame title
\setbeamerfont{section number projected}{% bold TOC bullet
  family=\rmfamily,series=\bfseries,size=\normalsize}
  
% two common fields in conference presentations
\newcommand\jointwork[1]{\def\insertjointwork{#1}}
\newcommand\conference[1]{\def\insertconference{#1}}

% Title page style
\setbeamertemplate{title page}{
\begin{tikzpicture}[remember picture, overlay]
\fill[UWWhite]
  ([yshift=30pt]current page.west) rectangle (current page.south east);

\fill[UWBlack]
  ([yshift=30pt]current page.west) rectangle (current page.north east);

\node[anchor=east] at ([yshift=-50pt,xshift=-15pt]current page.north east)
  {
  \includegraphics[width=0.4\linewidth]{./UniversityOfWaterloo_logo_vert_rev_rgb.png}};

\node[anchor=north west] at ([yshift=-70pt,xshift=15pt]current page.north west) (institute)
	{
	\parbox[t]{.78\paperwidth}{
    \usebeamerfont{institute}\usebeamercolor[fg]{institute}\large\bfseries\insertinstitute}
    };
    
\node[anchor=west] at ([yshift=-45pt,xshift=15pt]current page.north west) (author)
	{
	\parbox[t]{.78\paperwidth}{
    \usebeamerfont{author}\usebeamercolor[fg]{author}\Large\bfseries \insertauthor}
    };


    
\node[anchor=north] at ([yshift=15pt]current page.center) (title)
	{
	\parbox[t]{\textwidth}{\huge\bfseries\centering
	\usebeamerfont{title}\usebeamercolor[fg]{title}\inserttitle}
	};
    
\node[anchor=north] at ([yshift=-40pt]current page.center) (jointwork)
	{
	\parbox[t]{\paperwidth}{\bfseries\centering\insertjointwork}
	};
	
\node[anchor=north] at ([yshift=40pt]current page.south) (jointwork)
	{
	\parbox[t]{\paperwidth}{\centering\insertconference}
	};
\end{tikzpicture}
}

\setbeamertemplate{headline} % add navigation to headline
{%
  \begin{beamercolorbox}{section in head/foot}
    \vskip5pt\bfseries
    \insertnavigation{\paperwidth}
    \vskip2pt
  \end{beamercolorbox}%
}


\renewcommand*{\slideentry}[6]{} % no solid circle in headline

% three-parts footline, color determined in beamer template
\setbeamertemplate{footline}
{
	\leavevmode % vertical mode is ended and horizontal mode is entered. In vertical mode, TeX stacks horizontal boxes vertically, whereas in horizontal mode, they are taken as part of the text line. 
	\begin{beamercolorbox}[wd=.333333\paperwidth,ht=2.5ex,dp=1.125ex,
      leftskip=.3cm,rightskip=.3cm plus1fil]{author in head/foot}
		\usebeamerfont{author in head/foot}\insertshortauthor
    \end{beamercolorbox}%
    \begin{beamercolorbox}[wd=.333333\paperwidth,ht=2.5ex,dp=1.125ex,
      leftskip=.3cm,rightskip=.3cm plus1fil,center]{title in head/foot}
      {\usebeamerfont{title in head/foot}\insertshorttitle}
    \end{beamercolorbox}%
    \begin{beamercolorbox}[wd=.333333\paperwidth,ht=2.5ex,dp=1.125ex,
      leftskip=.3cm,rightskip=.3cm plus1fil]{institute in head/foot}
      \hfill    {\usebeamercolor[fg]{institute in head/foot}\insertshortinstitute}    
	\end{beamercolorbox}%
}

\setbeamertemplate{navigation symbols}{\bfseries\insertframenumber/\inserttotalframenumber}

\setbeamertemplate{sections/subsections in toc}[ball]

% make the itemize bullets pixelated >
\setbeamertemplate{itemize item}{
	\tikz{
		\draw[fill=spicy,draw=none] (0, 0) rectangle(0.075, 0.075);
		\draw[fill=spicy,draw=none] (0.075, 0.075) rectangle(0.15, 0.15);
		\draw[fill=spicy,draw=none] (0, 0.15) rectangle(0.075, 0.225);
	}
}

% make the subitems also pixelated >, but a little smaller and red
\setbeamertemplate{itemize subitem}{
	\tikz{
		\draw[fill=carrot,draw=none] (0, 0) rectangle(0.05, 0.05);
		\draw[fill=carrot,draw=none] (0.05, 0.05) rectangle(0.1, 0.1);
		\draw[fill=carrot,draw=none] (0, 0.1) rectangle(0.05, 0.15);
	}
}

\AtBeginEnvironment{block}{
	\setbeamertemplate{itemize item}{
		\tikz{
			\draw[fill=spicy,draw=none] (0, 0) rectangle(0.075, 0.075);
			\draw[fill=spicy,draw=none] (0.075, 0.075) rectangle(0.15, 0.15);
			\draw[fill=spicy,draw=none] (0, 0.15) rectangle(0.075, 0.225);
		}
	}

	\setbeamertemplate{itemize subitem}{
		\tikz{
			\draw[fill=carrot,draw=none] (0, 0) rectangle(0.05, 0.05);
			\draw[fill=carrot,draw=none] (0.05, 0.05) rectangle(0.1, 0.1);
			\draw[fill=carrot,draw=none] (0, 0.1) rectangle(0.05, 0.15);
		}
	}
}

\setbeamertemplate{blocks}[rounded][shadow=false]

\setbeamercovered{invisible}

\usefonttheme[onlymath]{serif} % change the math font theme

\AtBeginEnvironment{theorem}{%
  \setbeamercolor{block title}{bg=peppercorn, fg=pearl}
  \setbeamercolor{block body}{bg=parsnip, fg=spicy}
}

% set color scheme in different parts 
% please refer to beamer cheatsheet below for details
% http://www.cpt.univ-mrs.fr/~masson/latex/Beamer-appearance-cheat-sheet.pdf


\title[\textbf{Quantum Operators Modelling Measurement}]{Building Quantum Operators Modelling Measurement} % The short title appears at the bottom of every slide, the full title is only on the title page

\author[\textbf{Sid (s3bhatta@uwaterloo.ca)}]
{Siddhartha Bhattacharjee \\
Email: s3bhatta@uwaterloo.ca \\
Discord: booodaness} % Your name

\institute[\textbf{University of Waterloo, Faculty of Science}] % Your institution as it will appear on the bottom of every slide, may be shorthand to save space
{Mathematical Physics (1C)\\
University of Waterloo % Your institution for the title page
}

\jointwork{(Probably, that is)}
\conference{PhysClub Student Seminar, August 2, 2023}

%===== Uncomment the following if you wish to use references
%\usepackage[backend=bibtex,citestyle=authoryear-icomp,natbib,maxcitenames=1]{biblatex}
%\addbibresource{bibfile.bib}

% use this so appendices' page numbers do not count
\usepackage{appendixnumberbeamer}
\usepackage{cancel}

\newcommand{\pr}[0]{\text{pr}}
\newcommand{\tr}[0]{\text{tr}}

\begin{document}

% Title page, navigation surpressed, no page number
{
\beamertemplatenavigationsymbolsempty
\begin{frame}[plain]
\titlepage
\end{frame}
}

% TOC, navigation surpressed, no page number
{
\beamertemplatenavigationsymbolsempty
\defbeamertemplate*{headline}{miniframes theme no subsection no content}
{ \begin{beamercolorbox}{section in head/foot}
    \vskip\headheight
  \end{beamercolorbox}}
\begin{frame}{Outline} 
\tableofcontents 
\end{frame} 
}
\addtocounter{framenumber}{-2}

\section{Preliminaries}

\subsection{Axioms of Quantum Mechanics}
\begin{frame}{Axioms of Quantum Mechanics}
\begin{itemize}
\item In these slides, we will explicitly make use of the following axioms of quantum mechanics (QM):

\begin{block}{Axioms used}
\begin{enumerate}
\item The state $\left\lvert \Psi \right\rangle$ of a quantum system $\mathcal{S}$ can be represented by a vector in a separable Hilbert space $\mathbb{H}$.
\item Observables on $\mathcal{S}$ can be represented by self-adjoint linear operators $\mathbb{H} \to \mathbb{H}$ on the Hilbert space of states $\mathbb{H}$ of the system $\mathcal{S}$.
\end{enumerate}
\end{block}

\item Notice that we have not mentioned an axiom from the Hilbert space formulation of QM, commonly called the \textbf{Born rule}. In one form, it states that the probability that a quantum measurement of an observable $\widehat{A}$ makes a state $\left\lvert \Psi \right\rangle$ collapse to an eigenstate\footnote{a notion central to the Copenhagen interpretation of QM} $\left\lvert a_k \right\rangle$ is,

\begin{block}{Born rule}
$$\pr \left( \left\lvert a_k \right\rangle \right) = \left\langle \Psi \right. \left\lvert a_k \right\rangle \left\langle a_k \right. \left\lvert \Psi \right\rangle$$
\end{block}
\end{itemize}

We are, in fact, going to \emph{derive} the above principle from a simpler assumption!
\end{frame}

\subsection{Motivation}
\begin{frame}{Motivation}
\begin{itemize}
\item The motivation for this study begins by asking why the Born rule involves an expectation value. Before making the observation\footnote{For simplicity, we assume $\mathbb{H}$ has a dimension that is either finite or countably infinite.}, let us define the \textbf{expectation value} of a self-adjoint operator $\widehat{A} : \mathbb{H} \to \mathbb{H}$,

$$E \left( \widehat{A} \right) : = \sum_{k} \pr \left( \left\lvert a_k \right\rangle \right) a_k$$

where $\left\{ a_k \right\}$ is a normalized eigenbasis for $\mathbb{H}$, i.e., any state $\left\lvert \Psi \right\rangle \in \mathbb{H}$ can be written as a unique linear combination (over $\mathbb{C}$) of $\left\{ a_k \right\}$ and, for all $k$,

\begin{align*}
\widehat{A} \left\lvert a_k \right\rangle & = a_k \left\lvert a_k \right\rangle \\
\left\langle \left. a_k \right\rvert a_l \right\rangle & = \delta_{k l} := \begin{cases} 
1 & k = l \\
0 & k \neq l
\end{cases}
\end{align*}
\end{itemize}
\end{frame}

\begin{frame}{}
\begin{itemize}
\item As a consequence of the above, $\left\langle a_k \left. \right\rvert a_k \right\rangle = 1$ and we have,

\begin{align*}
a_k & = a_k \left\langle a_k \left. \right\rvert a_k \right\rangle \\
& = \left\langle a_k \left\lvert a_k \right\rvert a_k \right\rangle \\
& = \left\langle a_k \left\lvert \widehat{A} \right\rvert a_k \right\rangle
\end{align*}

Plugging this into the definition of the expectation value of $\widehat{A}$,

\begin{block}{Expectation values, without Born rule}
$$E \left( \widehat{A} \right) = \sum_{k} \pr \left( \left\lvert a_k \right\rangle \right) \left\langle a_k \left\lvert \widehat{A} \right\rvert a_k \right\rangle$$
\end{block}

Using the Born rule,

\begin{align*}
E \left( \widehat{A} \right) = \sum_{k} \pr \left( \left\lvert a_k \right\rangle \right) \left\langle a_k \left\lvert \widehat{A} \right\rvert a_k \right\rangle & = \sum_k \left\langle \Psi \right. \left\lvert a_k \right\rangle \left\langle a_k \right. \left\lvert \Psi \right\rangle \left\langle a_k \left\lvert \widehat{A} \right\rvert a_k \right\rangle \\
& = \sum_k \sum_l \left\langle \Psi \right. \left\lvert a_k \right\rangle \left\langle a_k \left\lvert \widehat{A} \right\rvert a_l \right\rangle \left\langle a_l \right. \left\lvert \Psi \right\rangle \\
& = \left\langle \Psi \left\lvert \widehat{A} \right\rvert \Psi \right\rangle
\end{align*}
\end{itemize}
\end{frame}

\begin{frame}{}
\begin{itemize}
\item Therefore, the Born rule simplifies the expression for the expectation value of a quantum operator. 

\begin{block}{Expectation values, with Born rule}
$$E \left( \widehat{A} \right) = \left\langle \Psi \left\lvert \widehat{A} \right\rvert \Psi \right\rangle$$
\end{block}


\item It follows that the Born rule itself hides the expectation value of a projection operators corresponding to eigenstates:

\begin{align*}
\pr \left( \left\lvert a_k \right\rangle \right) & = \left\langle \Psi \right. \left\lvert a_k \right\rangle \left\langle a_k \right. \left\lvert \Psi \right\rangle \\
& = \left\langle \Psi \right\rvert \left\lvert a_k \right\rangle \left\langle a_k \right\rvert \left\lvert \Psi \right\rangle \\
& = E \left( \left\lvert a_k \right\rangle \left\langle a_k \right\rvert \right)
\end{align*}

\begin{block}{Born rule, with expectation values}
$$\pr \left( \left\lvert a_k \right\rangle \right) = E \left( \left\lvert a_k \right\rangle \left\langle a_k \right\rvert \right)$$
\end{block}
\end{itemize}
\end{frame}

\begin{frame}{}
\begin{itemize}
\item Equipped with the above ideas, we note,

\begin{align*}
E \left( \widehat{I} \right) & = \left\langle \Psi \left\lvert \widehat{I} \right\rvert \Psi \right\rangle \\
& = \left\langle \Psi \left\lvert \right. \Psi \right\rangle
\end{align*}

But,

\begin{align*}
E \left( \widehat{I} \right) : & = \sum_{k} \pr \left( \left\lvert a_k \right\rangle \right) \left\langle a_k \left\lvert \widehat{I} \right\rvert a_k \right\rangle \\
& = \sum_k \pr \left( \left\lvert a_k \right\rangle \right) \left\langle a_k \left. \right\rvert a_k \right\rangle \\
& = \sum_k \pr \left( \left\lvert a_k \right\rangle \right) \\
: & = 1
\end{align*}

Therefore, we have,

\begin{block}{Normalization}
$$\left\langle \Psi \left\lvert \right. \Psi \right\rangle = 1$$
\end{block}
\end{itemize}
\end{frame}

\begin{frame}{}
\begin{itemize}
\item However, for the purposes of these slides, the above statement is not necessary. We could have, for instance, modified the Born rule without loss or gain of theory as,

\begin{block}{Born rule, with explicit normalization}
$$\pr \left( \left\lvert a_k \right\rangle \right) = \frac{1}{\left\langle \Psi \left\lvert \right. \Psi \right\rangle} \left\langle \Psi \right. \left\lvert a_k \right\rangle \left\langle a_k \right. \left\lvert \Psi \right\rangle$$
\end{block}

Explicit normalization of states then becomes unnecessary as the above rule is invariant under normalization of the form $\displaystyle{\left\lvert \Psi \right\rangle \to \frac{1}{\left\langle \Psi \left\lvert \right. \Psi \right\rangle} \left\lvert \Psi \right\rangle}$.

\item In general, the idea is that scaling states by \emph{any} complex number should not change physics\footnote{In \emph{The Principles of Quantum Mechanics}, Paul Dirac gives great attention to this point and how it is related to the idea that eigenstates matter only up to scale as orthogonality of states is a physical distinction and scaling does not disturb orthogonality.}; this idea will be formalized later on.
\end{itemize}
\end{frame}

\subsection{Density Operators}
\begin{frame}{Pure and Mixed Quantum States}
\begin{itemize}
\item Recall that a quantum system has a state $\left\lvert \Psi \right\rangle$ belonging to a [separable] Hilbert space $\mathcal{H}$. But there is more to a state than this notion, as follows.

\item A quantum state $\left\lvert \Psi \right\rangle$ is \textbf{pure} if it is described by a \emph{single} ket, say $\left\lvert \Psi_1 \right\rangle \in \mathcal{H}$,

$$\left\lvert \Psi \right\rangle = \left\lvert \Psi_1 \right\rangle$$

\item A quantum state $\left\lvert \Psi \right\rangle$ is \textbf{mixed} if it is possibly described by \emph{multiple} kets, say $\left\lvert \Psi_1 \right\rangle, \left\lvert \Psi_2 \right\rangle, \dots \left\lvert \Psi_N \right\rangle \in \mathcal{H}$. 

The probability of $\left\lvert \Psi \right\rangle$ being described by a given state $\left\lvert \Psi_\alpha \right\rangle$ can be described by a probability map,

\begin{align*}
\pr : \left\lvert \Psi_\alpha \right\rangle & \to \text{pr} \left( \left\lvert \Psi_\alpha \right\rangle \right) \in \left[ 0, 1 \right] \\
\sum_{\alpha} \pr \left( \left\lvert \Psi_\alpha \right\rangle \right) & = 1
\end{align*}
\end{itemize}
\end{frame}

\begin{frame}{Density Operators}
\begin{itemize}
\item In general, the information contained in the possible state(s) of a quantum system are packed into what is called its \textbf{density operator} $\widehat{\rho}$,

\begin{block}{Density operators}
$$\widehat{\rho} : = \sum_{\alpha} \pr \left( \left\lvert \Psi_\alpha \right\rangle \right) \left\lvert \Psi_\alpha \right\rangle \left\langle \Psi_\alpha \right\rvert$$
\end{block}

For a pure state $\left\lvert \Psi \right\rangle$, the density operator is simply $\left\lvert \Psi \right\rangle \left\langle \Psi \right\rvert$.

\item For future use, we define the trace of a linear operator,

\begin{block}{Trace of linear operators}
$$\tr \left( \widehat{A} \right) : = \sum_k \left\langle a_k \left\lvert \widehat{A} \right\rvert a_k \right\rangle = \sum_k a_k$$
\end{block}

We will soon use these constructions to simplify the expression for the expectation value of a linear operator.
\end{itemize}
\end{frame}

\begin{frame}{Properties of Trace}
\begin{itemize}
\item Firstly, trace is a \textbf{linear} operation as,

\begin{align*}
\tr \left( c \widehat{A} \right) & : = \sum_k \left\langle a_k \left\lvert c \widehat{A} \right\rvert a_k \right\rangle \\
& = c \sum_k \left\langle a_k \left\lvert \widehat{A} \right\rvert a_k \right\rangle \\
& = c \: \tr \left( \widehat{A} \right) \\
\tr \left( \widehat{A} + \widehat{B} \right) : & = \sum_k \left\langle a_k \left\lvert \left( \widehat{A} + \widehat{B} \right) \right\rvert a_k \right\rangle \\
& = \sum_k \left[ \left\langle a_k \left\lvert \widehat{A} \right\rvert a_k \right\rangle + \left\langle a_k \left\lvert \widehat{B} \right\rvert a_k \right\rangle \right] \\
& = \sum_k \left\langle a_k \left\lvert \widehat{A} \right\rvert a_k \right\rangle + \sum_k \left\langle a_k \left\lvert \widehat{B} \right\rvert a_k \right\rangle \\
& = \tr \left( \widehat{A} \right) + \tr \left( \widehat{B} \right)
\end{align*}
\end{itemize}
\end{frame}

\begin{frame}{}
\begin{itemize}
\item Secondly, trace is a \emph{symmetric} operation,

$$\tr \left( \widehat{A} \widehat{B} \right) : = \sum_m \left\langle a_m \left\lvert \widehat{A} \widehat{B} \right\rvert a_m \right\rangle$$

\begin{align*}
& = \sum_m \left\langle a_m \right\rvert \left( \sum_i \sum_j \left\langle a_i \left\lvert \widehat{A} \right\vert a_j \right\rangle \left\lvert a_i \right\rangle \left\langle a_j \right\rvert \right) \left( \sum_k \sum_l \left\langle a_k \left\lvert \widehat{B} \right\vert a_l \right\rangle \left\lvert a_k \right\rangle \left\langle a_l \right\rvert \right) \left\lvert a_m \right\rangle \\
& = \sum_m \sum_i \sum_j \sum_k \sum_l \left\langle a_i \left\lvert \widehat{A} \right\vert a_j \right\rangle \left\langle a_k \left\lvert \widehat{B} \right\vert a_l \right\rangle \left\langle a_m \right\rvert \left. a_i \right\rangle \left\langle a_j \right\rvert \left. a_k \right\rangle \left\langle a_l \right\rvert \left. a_m \right\rangle \\
& = \sum_m \cancel{\sum_i} \sum_j \cancel{\sum_k} \cancel{\sum_l} \left\langle a_{\cancel{i}} \left\lvert \widehat{A} \right\vert a_j \right\rangle \left\langle a_{\cancel{k}} \left\lvert \widehat{B} \right\vert a_{\cancel{l}} \right\rangle \delta_{m \cancel{i}} \delta_{j \cancel{k}} \delta_{\cancel{l} m} \\
& = \sum_m \sum_j \left\langle a_{m} \left\lvert \widehat{A} \right\vert a_j \right\rangle \left\langle a_{j} \left\lvert \widehat{B} \right\vert a_{m} \right\rangle \\
& = \sum_m \sum_j \left\langle a_{j} \left\lvert \widehat{B} \right\vert a_{m} \right\rangle \left\langle a_{m} \left\lvert \widehat{A} \right\vert a_j \right\rangle \\
& = \sum_m \sum_j \left\langle a_{m} \left\lvert \widehat{B} \right\vert a_{j} \right\rangle \left\langle a_{j} \left\lvert \widehat{A} \right\vert a_{m} \right\rangle \\
& = \tr \left( \widehat{B} \widehat{A} \right)
\end{align*}
\end{itemize}
\end{frame}

\begin{frame}{}
\begin{itemize}
\item To summarize,

\begin{block}{Properties of trace}
\begin{enumerate}
\item Linearity

\begin{align*}
\tr \left( c \widehat{A} \right) & = c \tr \left( \widehat{A} \right) \\
\tr \left( \widehat{A} + \widehat{B} \right) & = \tr \left( \widehat{A} \right) + \tr \left( \widehat{B} \right)
\end{align*}

\item Symmetry

$$\tr \left( \widehat{A} \widehat{B} \right) = \tr \left( \widehat{B} \widehat{A} \right)$$
\end{enumerate}
\end{block}

\item As a corollary,

\begin{align*}
\tr \left( \left( \widehat{A}_1 \dots A_{M-1} \right) \widehat{A}_M \right) & = \tr \left( \widehat{A}_M \left( \widehat{A}_1 \dots \widehat{A}_{M-1} \right) \right) = \dots
\end{align*}

\begin{block}{Cyclicity}
$$\tr \left( \widehat{A}_1 \dots \widehat{A}_M \right) = \tr \left( \widehat{A}_M \widehat{A}_1 \dots \widehat{A}_{M-1} \right) = \tr \left( \widehat{A}_{M-1} \widehat{A}_M \widehat{A}_1 \dots \widehat{A}_{M-2} \right) = \dots$$
\end{block}
\end{itemize}
\end{frame}

\begin{frame}{Expectation Values Using Density Operators}
\begin{itemize}
\item For a pure quantum state $\left\lvert \Psi \right\rangle$,

\begin{align*}
E \left( \widehat{A} \right) = \left\langle \Psi \left\lvert \widehat{A} \right\rvert \Psi \right\rangle & = \left( \sum_k \left\langle \Psi \right. \left\lvert a_k \right\rangle \left\langle a_k \right\rvert \right) \widehat{A} \left( \sum_l \left\lvert a_l \right\rangle \left\langle a_l \right\rvert \left. \Psi \right\rangle \right) \\
& = \sum_k \sum_l \left\langle \Psi \right. \left\lvert a_k \right\rangle \left\langle a_l \right. \left\lvert \Psi \right\rangle \left\langle a_k \left\lvert \widehat{A} \right\rvert a_l \right\rangle \\
& = \sum_k \sum_l \left\langle \Psi \right. \left\lvert a_k \right\rangle \left\langle a_l \right. \left\lvert \Psi \right\rangle \left\langle a_k \left\lvert a_k \right\rvert a_k \right\rangle \\
& = \sum_k \sum_l \left\langle \Psi \right. \left\lvert a_k \right\rangle \left\langle a_l \right. \left\lvert \Psi \right\rangle \left\langle a_k \right\rvert \left. a_l \right\rangle a_k \\
& = \tr \left( \sum_k \sum_l \left\langle \Psi \right. \left\lvert a_k \right\rangle \left\langle a_l \right. \left\lvert \Psi \right\rangle \left\lvert a_l \right\rangle \left\langle a_k \right\rvert a_k \right) \\
& = \tr \left[ \left( \sum_k \left\langle \Psi \right. \left\lvert a_k \right\rangle \left\langle a_k \right\rvert \right) \left( \sum_l \left\lvert a_l \right\rangle \left\langle a_l \right\rvert \left. \Psi \right\rangle \right) \widehat{A} \right] \\
& = \tr \left( \left\lvert \Psi \right\rangle \left\langle \Psi \right\rvert \widehat{A} \right)
\end{align*}
\end{itemize}
\end{frame}

\begin{frame}{}
\begin{itemize}
\item Let us redefine the expectation value of a linear operator $\widehat{A}$ taking into account mixed states,

$$E \left( \widehat{A} \right) := \sum_{\alpha} \pr \left( \left\lvert \Psi_\alpha \right\rangle \right) E_\alpha \left( \widehat{A} \right)$$

where $E_\alpha \left( \widehat{A} \right)$ is the previously-defined notion of expectation values, which holds for pure states $\left\lvert \Psi \right\rangle = \left\lvert \Psi_\alpha \right\rangle$.

\begin{align*}
E \left( \widehat{A} \right) & = \sum_{\alpha} \pr \left( \left\lvert \Psi_\alpha \right\rangle \right) \tr \left( \left\lvert \Psi \right\rangle \left\langle \Psi \right\rangle \widehat{A} \right) \\
& = \sum_{\alpha} \tr \left( \pr \left( \left\lvert \Psi_\alpha \right\rangle \right) \left\lvert \Psi \right\rangle \left\langle \Psi \right\rangle \widehat{A} \right) \\
& = \tr \left( \sum_{\alpha} \pr \left( \left\lvert \Psi_\alpha \right\rangle \right) \left\lvert \Psi \right\rangle \left\langle \Psi \right\rangle \widehat{A} \right) \\
& = \tr \left( \widehat{\rho} \widehat{A} \right)
\end{align*}

\begin{block}{Expectation values, with Born rule, using density operators}
$$E \left( \widehat{A} \right) = \tr \left( \widehat{\rho} \widehat{A} \right)$$
\end{block}
\end{itemize}
\end{frame}

\begin{frame}{}
\begin{itemize}
\item This leads to the following corollary,

\begin{block}{Born rule, with density operators}
$$\pr \left( \left\lvert a_k \right\rangle \right) = \tr \left( \widehat{\rho} \left\lvert a_k \right\rangle \left\langle a_k \right\rvert \right)$$
\end{block}
\end{itemize}
\end{frame}

%% References
%\appendix
%\begin{frame}[allowframebreaks]
%        \frametitle{References}
%        \printbibliography
%\end{frame}
\end{document}