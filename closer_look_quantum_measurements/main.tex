\documentclass[9pt,handout]{beamer}

% beamerthemeFeng.sty
% style file for beamer presentation

% tikz is used to ``draw'' title page and other templates in beamer
\usepackage{tikz,etoolbox}
\usetikzlibrary{shapes,arrows}

\definecolor{UWBlack}{HTML}{000000}
\definecolor{UWWhite}{HTML}{FFFFFF}


\definecolor{UWMathPinkL1}{HTML}{FFBEEF}
\definecolor{UWMathPinkL2}{HTML}{FF63AA}
\definecolor{UWMathPinkL3}{HTML}{DF2498}
\definecolor{UWMathPinkL4}{HTML}{C60078}
\definecolor{UWGrayL1}{HTML}{DFDFDF}
\definecolor{UWGrayL2}{HTML}{A2A2A2}
\definecolor{UWGrayL3}{HTML}{787878}
\definecolor{UWGrayL4}{HTML}{000000}
\definecolor{UWGoldL1}{HTML}{FFFFAA}
\definecolor{UWGoldL2}{HTML}{FFEA3D}
\definecolor{UWGoldL3}{HTML}{FFD54F}
\definecolor{UWGoldL4}{HTML}{E4B429}

\definecolor{carrot}{HTML}{EE693F}
\definecolor{ivory}{HTML}{F1F3CE}
\definecolor{emerald}{HTML}{265C00}
\definecolor{turquise}{HTML}{5BC8AC}
\definecolor{peacockblue}{HTML}{1E656D}
\definecolor{spicy}{HTML}{B51D0A}
\definecolor{bluegreen}{HTML}{5F968E}
\definecolor{rust}{HTML}{9B4F0F}
\definecolor{burntorange}{HTML}{DE7A22}
\definecolor{sea}{HTML}{20948B}
\definecolor{lagoon}{HTML}{6AB187}


% Set colors for different components in a slide
\setbeamercolor{background canvas}{bg=UWWhite}
\setbeamercolor{author}{fg=UWGoldL3}
\setbeamercolor{institute}{fg=UWMathPinkL3}
\setbeamercolor{title}{fg=UWGrayL4}
\setbeamercolor{section in head/foot}{bg=UWBlack, fg=UWGoldL3}
\setbeamercolor{author in head/foot}{fg=UWGoldL3, bg=UWBlack}
\setbeamercolor{title in head/foot}{fg=UWBlack,bg=UWGoldL3}
\setbeamercolor{institute in head/foot}{fg=UWGoldL3, bg=UWBlack}
\setbeamercolor{navigation symbols}{fg=UWBlack}
\setbeamercolor{normal text}{fg=UWGrayL3}
\setbeamercolor{section in toc}{fg=emerald}
\setbeamercolor{subsection in toc}{fg=bluegreen}
\setbeamercolor{frametitle}{fg=UWMathPinkL2, bg=UWGrayL1}
\setbeamercolor{block title}{bg=emerald, fg=ivory}
\setbeamercolor{block body}{bg=peacockblue!20, fg=peacockblue}
\setbeamercolor{section number projected}{bg=turquise,fg=black}
\setbeamercolor{block title example}{fg=rust,
	bg= sea!40}
\setbeamercolor{block body example}{fg= burntorange,
	bg= lagoon!20}

\setbeamerfont{frametitle}{series=\bfseries} % bold frame title
\setbeamerfont{section number projected}{% bold TOC bullet
  family=\rmfamily,series=\bfseries,size=\normalsize}
  
% two common fields in conference presentations
\newcommand\jointwork[1]{\def\insertjointwork{#1}}
\newcommand\conference[1]{\def\insertconference{#1}}

% Title page style
\setbeamertemplate{title page}{
\begin{tikzpicture}[remember picture, overlay]
\fill[UWWhite]
  ([yshift=30pt]current page.west) rectangle (current page.south east);

\fill[UWBlack]
  ([yshift=30pt]current page.west) rectangle (current page.north east);

\node[anchor=east] at ([yshift=-50pt,xshift=-15pt]current page.north east)
  {
  \includegraphics[width=0.4\linewidth]{./UniversityOfWaterloo_logo_vert_rev_rgb.png}};

\node[anchor=north west] at ([yshift=-70pt,xshift=15pt]current page.north west) (institute)
	{
	\parbox[t]{.78\paperwidth}{
    \usebeamerfont{institute}\usebeamercolor[fg]{institute}\large\bfseries\insertinstitute}
    };
    
\node[anchor=west] at ([yshift=-45pt,xshift=15pt]current page.north west) (author)
	{
	\parbox[t]{.78\paperwidth}{
    \usebeamerfont{author}\usebeamercolor[fg]{author}\Large\bfseries \insertauthor}
    };


    
\node[anchor=north] at ([yshift=15pt]current page.center) (title)
	{
	\parbox[t]{\textwidth}{\huge\bfseries\centering
	\usebeamerfont{title}\usebeamercolor[fg]{title}\inserttitle}
	};
    
\node[anchor=north] at ([yshift=-40pt]current page.center) (jointwork)
	{
	\parbox[t]{\paperwidth}{\bfseries\centering\insertjointwork}
	};
	
\node[anchor=north] at ([yshift=40pt]current page.south) (jointwork)
	{
	\parbox[t]{\paperwidth}{\centering\insertconference}
	};
\end{tikzpicture}
}

\setbeamertemplate{headline} % add navigation to headline
{%
  \begin{beamercolorbox}{section in head/foot}
    \vskip5pt\bfseries
    \insertnavigation{\paperwidth}
    \vskip2pt
  \end{beamercolorbox}%
}


\renewcommand*{\slideentry}[6]{} % no solid circle in headline

% three-parts footline, color determined in beamer template
\setbeamertemplate{footline}
{
	\leavevmode % vertical mode is ended and horizontal mode is entered. In vertical mode, TeX stacks horizontal boxes vertically, whereas in horizontal mode, they are taken as part of the text line. 
	\begin{beamercolorbox}[wd=.333333\paperwidth,ht=2.5ex,dp=1.125ex,
      leftskip=.3cm,rightskip=.3cm plus1fil]{author in head/foot}
		\usebeamerfont{author in head/foot}\insertshortauthor
    \end{beamercolorbox}%
    \begin{beamercolorbox}[wd=.333333\paperwidth,ht=2.5ex,dp=1.125ex,
      leftskip=.3cm,rightskip=.3cm plus1fil,center]{title in head/foot}
      {\usebeamerfont{title in head/foot}\insertshorttitle}
    \end{beamercolorbox}%
    \begin{beamercolorbox}[wd=.333333\paperwidth,ht=2.5ex,dp=1.125ex,
      leftskip=.3cm,rightskip=.3cm plus1fil]{institute in head/foot}
      \hfill    {\usebeamercolor[fg]{institute in head/foot}\insertshortinstitute}    
	\end{beamercolorbox}%
}

\setbeamertemplate{navigation symbols}{\bfseries\insertframenumber/\inserttotalframenumber}

\setbeamertemplate{sections/subsections in toc}[ball]

% make the itemize bullets pixelated >
\setbeamertemplate{itemize item}{
	\tikz{
		\draw[fill=spicy,draw=none] (0, 0) rectangle(0.075, 0.075);
		\draw[fill=spicy,draw=none] (0.075, 0.075) rectangle(0.15, 0.15);
		\draw[fill=spicy,draw=none] (0, 0.15) rectangle(0.075, 0.225);
	}
}

% make the subitems also pixelated >, but a little smaller and red
\setbeamertemplate{itemize subitem}{
	\tikz{
		\draw[fill=carrot,draw=none] (0, 0) rectangle(0.05, 0.05);
		\draw[fill=carrot,draw=none] (0.05, 0.05) rectangle(0.1, 0.1);
		\draw[fill=carrot,draw=none] (0, 0.1) rectangle(0.05, 0.15);
	}
}

\AtBeginEnvironment{block}{
	\setbeamertemplate{itemize item}{
		\tikz{
			\draw[fill=spicy,draw=none] (0, 0) rectangle(0.075, 0.075);
			\draw[fill=spicy,draw=none] (0.075, 0.075) rectangle(0.15, 0.15);
			\draw[fill=spicy,draw=none] (0, 0.15) rectangle(0.075, 0.225);
		}
	}

	\setbeamertemplate{itemize subitem}{
		\tikz{
			\draw[fill=carrot,draw=none] (0, 0) rectangle(0.05, 0.05);
			\draw[fill=carrot,draw=none] (0.05, 0.05) rectangle(0.1, 0.1);
			\draw[fill=carrot,draw=none] (0, 0.1) rectangle(0.05, 0.15);
		}
	}
}

\setbeamertemplate{blocks}[rounded][shadow=false]

\setbeamercovered{invisible}

\usefonttheme[onlymath]{serif} % change the math font theme

\AtBeginEnvironment{theorem}{%
  \setbeamercolor{block title}{bg=peppercorn, fg=pearl}
  \setbeamercolor{block body}{bg=parsnip, fg=spicy}
}

% set color scheme in different parts 
% please refer to beamer cheatsheet below for details
% http://www.cpt.univ-mrs.fr/~masson/latex/Beamer-appearance-cheat-sheet.pdf


\title[\textbf{A Closer Look at Quantum Measurements}]{A Closer Look at Quantum Measurements} % The short title appears at the bottom of every slide, the full title is only on the title page

\author[\textbf{Sid (s3bhatta@uwaterloo.ca)}]
{Siddhartha Bhattacharjee \\
Email: s3bhatta@uwaterloo.ca \\
Discord: booodaness} % Your name

\institute[\textbf{University of Waterloo, Faculty of Science}] % Your institution as it will appear on the bottom of every slide, may be shorthand to save space
{Mathematical Physics (1C)\\
University of Waterloo % Your institution for the title page
}

\jointwork{}
\conference{PhysClub Student Seminar, August 2, 2023}

%===== Uncomment the following if you wish to use references
%\usepackage[backend=bibtex,citestyle=authoryear-icomp,natbib,maxcitenames=1]{biblatex}
%\addbibresource{bibfile.bib}

% use this so appendices' page numbers do not count
\usepackage{appendixnumberbeamer}
\usepackage{cancel}
\usepackage{hyperref}
\hypersetup{
    colorlinks=true,
    linkcolor=.,
    filecolor=.,      
    urlcolor=blue,
    }
\usepackage{tikz-cd}

\newcommand{\pr}[0]{\text{pr}}
\newcommand{\tr}[0]{\text{tr}}

\begin{document}

% Title page, navigation surpressed, no page number
{
\beamertemplatenavigationsymbolsempty
\begin{frame}[plain]
\titlepage
\end{frame}
}

\begin{frame}{Abstract}
In this study, we investigate the nature of quantum measurement by exploring the relationships between the Born rule, expectation values, density operators and such notions from first principles. 
  
This reveals a generalized form of the Born rule. Moreover, the usual form of the Born rule can be sufficiently characterized by an explicit modelling of wavefunction collapse and measurement-invariant density operators.
\end{frame}

% TOC, navigation surpressed, no page number
{
\beamertemplatenavigationsymbolsempty
\defbeamertemplate*{headline}{miniframes theme no subsection no content}
{ \begin{beamercolorbox}{section in head/foot}
    \vskip\headheight
  \end{beamercolorbox}}
\begin{frame}{Outline} 
\tableofcontents 
\end{frame} 
}
\addtocounter{framenumber}{-2}

\section{Preliminaries}

\subsection{Axioms of Quantum Mechanics}
\begin{frame}{Axioms of Quantum Mechanics}
\begin{itemize}
\item In these slides, we will explicitly make use of the following axioms of quantum mechanics (QM):

\begin{block}{Axioms necessarily used}
\begin{enumerate}
\item The state $\left\lvert \Psi \right\rangle$ of a quantum system $\mathcal{S}$ can be represented by a vector in a separable Hilbert space $\mathbb{H}$.
\item Observables on $\mathcal{S}$ can be represented by self-adjoint linear operators $\mathbb{H} \to \mathbb{H}$ on the Hilbert space of states $\mathbb{H}$ of the system $\mathcal{S}$.
\end{enumerate}
\end{block}

\item Notice that we have not mentioned an axiom from the Hilbert space formulation of QM, commonly called the \textbf{Born rule}. In one form, it states that the probability that a quantum measurement of an observable $\widehat{A}$ makes a state $\left\lvert \Psi \right\rangle$ collapse to an eigenstate\footnote{a notion central to the Copenhagen interpretation of QM} $\left\lvert a_k \right\rangle$ is,

\begin{block}{Born rule}
$$\pr \left( \left\lvert a_k \right\rangle \right) = \left\langle \Psi \right. \left\lvert a_k \right\rangle \left\langle a_k \right. \left\lvert \Psi \right\rangle$$
\end{block}
\end{itemize}

We are, in fact, going to \emph{derive} the above principle from a simpler assumption!
\end{frame}

\subsection{Motivation}
\begin{frame}{Motivation}
\begin{itemize}
\item The motivation for this study begins by asking why the Born rule involves an expectation value. Before making the observation\footnote{For simplicity, we assume $\mathbb{H}$ has a dimension that is either finite or countably infinite.}, let us define the \textbf{expectation value} of a self-adjoint operator $\widehat{A} : \mathbb{H} \to \mathbb{H}$,

$$E \left( \widehat{A} \right) : = \sum_{k} \pr \left( \left\lvert a_k \right\rangle \right) a_k$$

where $\left\{ a_k \right\}$ is a normalized eigenbasis for $\mathbb{H}$, i.e., any state $\left\lvert \Psi \right\rangle \in \mathbb{H}$ can be written as a unique linear combination (over $\mathbb{C}$) of $\left\{ a_k \right\}$ and, for all $k$,

\begin{align*}
\widehat{A} \left\lvert a_k \right\rangle & = a_k \left\lvert a_k \right\rangle \\
\left\langle \left. a_k \right\rvert a_l \right\rangle & = \delta_{k l} := \begin{cases} 
1 & k = l \\
0 & k \neq l
\end{cases}
\end{align*}
\end{itemize}
\end{frame}

\begin{frame}{}
\begin{itemize}
\item As a consequence of the above, $\left\langle a_k \left. \right\rvert a_k \right\rangle = 1$ and we have,

\begin{align*}
a_k & = a_k \left\langle a_k \left. \right\rvert a_k \right\rangle \\
& = \left\langle a_k \left\lvert a_k \right\rvert a_k \right\rangle \\
& = \left\langle a_k \left\lvert \widehat{A} \right\rvert a_k \right\rangle
\end{align*}

Plugging this into the definition of the expectation value of $\widehat{A}$,

\begin{block}{Expectation values, without Born rule}
$$E \left( \widehat{A} \right) = \sum_{k} \pr \left( \left\lvert a_k \right\rangle \right) \left\langle a_k \left\lvert \widehat{A} \right\rvert a_k \right\rangle$$
\end{block}

Using the Born rule,

\begin{align*}
E \left( \widehat{A} \right) = \sum_{k} \pr \left( \left\lvert a_k \right\rangle \right) \left\langle a_k \left\lvert \widehat{A} \right\rvert a_k \right\rangle & = \sum_k \left\langle \Psi \right. \left\lvert a_k \right\rangle \left\langle a_k \right. \left\lvert \Psi \right\rangle \left\langle a_k \left\lvert \widehat{A} \right\rvert a_k \right\rangle \\
& = \sum_k \sum_l \left\langle \Psi \right. \left\lvert a_k \right\rangle \left\langle a_k \left\lvert \widehat{A} \right\rvert a_l \right\rangle \left\langle a_l \right. \left\lvert \Psi \right\rangle \\
& = \left\langle \Psi \left\lvert \widehat{A} \right\rvert \Psi \right\rangle
\end{align*}
\end{itemize}
\end{frame}

\begin{frame}{}
\begin{itemize}
\item Therefore, the Born rule simplifies the expression for the expectation value of a quantum operator. 

\begin{block}{Expectation values, with Born rule}
$$E \left( \widehat{A} \right) = \left\langle \Psi \left\lvert \widehat{A} \right\rvert \Psi \right\rangle$$
\end{block}


\item It follows that the Born rule itself hides the expectation value of projectors corresponding to eigenstates:

\begin{align*}
\pr \left( \left\lvert a_k \right\rangle \right) & = \left\langle \Psi \right. \left\lvert a_k \right\rangle \left\langle a_k \right. \left\lvert \Psi \right\rangle \\
& = \left\langle \Psi \right\rvert \left\lvert a_k \right\rangle \left\langle a_k \right\rvert \left\lvert \Psi \right\rangle \\
& = E \left( \left\lvert a_k \right\rangle \left\langle a_k \right\rvert \right)
\end{align*}

\begin{block}{Born rule, with expectation values}
$$\pr \left( \left\lvert a_k \right\rangle \right) = E \left( \left\lvert a_k \right\rangle \left\langle a_k \right\rvert \right)$$
\end{block}
\end{itemize}
\end{frame}

\begin{frame}{}
\begin{itemize}
\item In fact, the above statement can be derived from first principles using the definition of expectation values. 

It can be proved that the eigenbasis for a projector $\left\lvert a_k \right\rangle \left\langle a_k \right\rvert$ is the same as that of the corresponding operator $\widehat{A}$. Thus,

\begin{align*}
E \left( \left\lvert a_k \right\rangle \left\langle a_k \right\rvert \right) & = \sum_l \pr \left( \left\lvert a_l \right\rangle \right) \left\langle a_l \right\rvert \left. a_k \right\rangle \left\langle a_k \right\rvert \left. a_l \right\rangle \\
& = \sum_l \pr \left( \left\lvert a_l \right\rangle \right) \delta_{l k} \delta_{k l} \\
& = \cancel{\sum_l} \pr \left( \left\lvert a_l \right\rangle \right) \delta_{k \cancel{l}} \\
& = \pr \left( \left\lvert a_k \right\rangle \right)
\end{align*}

\item Therefore, the above statement is independent of the Born rule itself.
\end{itemize}
\end{frame}

\begin{frame}{}
\begin{itemize}
\item Equipped with the above ideas, we note,

\begin{align*}
E \left( \widehat{I} \right) & = \left\langle \Psi \left\lvert \widehat{I} \right\rvert \Psi \right\rangle \\
& = \left\langle \Psi \left\lvert \right. \Psi \right\rangle
\end{align*}

But,

\begin{align*}
E \left( \widehat{I} \right) : & = \sum_{k} \pr \left( \left\lvert a_k \right\rangle \right) \left\langle a_k \left\lvert \widehat{I} \right\rvert a_k \right\rangle \\
& = \sum_k \pr \left( \left\lvert a_k \right\rangle \right) \left\langle a_k \left. \right\rvert a_k \right\rangle \\
& = \sum_k \pr \left( \left\lvert a_k \right\rangle \right) \\
: & = 1
\end{align*}

Therefore, we have,

\begin{block}{Normalization}
$$\left\langle \Psi \left\lvert \right. \Psi \right\rangle = 1$$
\end{block}
\end{itemize}
\end{frame}

\begin{frame}{}
\begin{itemize}
\item However, for the purposes of these slides, the above statement is not necessary. We could have, for instance, modified the Born rule without loss or gain of theory as,

\begin{block}{Born rule, with explicit normalization}
$$\pr \left( \left\lvert a_k \right\rangle \right) = \frac{1}{\left\langle \Psi \left\lvert \right. \Psi \right\rangle} \left\langle \Psi \right. \left\lvert a_k \right\rangle \left\langle a_k \right. \left\lvert \Psi \right\rangle$$
\end{block}

Explicit normalization of states then becomes unnecessary as the above rule is invariant under normalization of the form $\displaystyle{\left\lvert \Psi \right\rangle \to \frac{1}{\left[ \left\langle \Psi \left\lvert \right. \Psi \right\rangle \right]^{1/2}} \left\lvert \Psi \right\rangle}$.

\item In general, the idea is that scaling states by \emph{any} complex number should not change physics\footnote{In \emph{The Principles of Quantum Mechanics}, Paul Dirac gives great attention to this point and how it is related to the idea that eigenstates matter only up to scale as orthogonality of states is a physical distinction and scaling does not disturb orthogonality.}; this idea will be formalized later on.
\end{itemize}
\end{frame}

\subsection{Density Operators}
\begin{frame}{Pure and Mixed Quantum States}
\begin{itemize}
\item Recall that a quantum system has a state $\left\lvert \Psi \right\rangle$ belonging to a [separable] Hilbert space $\mathcal{H}$. But there is more to a state than this notion, as follows.

\item A quantum state $\left\lvert \Psi \right\rangle$ is \textbf{pure} if it is described by a \emph{single} ket, say $\left\lvert \Psi_1 \right\rangle \in \mathcal{H}$,

$$\left\lvert \Psi \right\rangle = \left\lvert \Psi_1 \right\rangle$$

\item A quantum state $\left\lvert \Psi \right\rangle$ is \textbf{mixed} if it is possibly described by \emph{multiple} kets, say $\left\lvert \Psi_1 \right\rangle, \left\lvert \Psi_2 \right\rangle, \dots \left\lvert \Psi_N \right\rangle \in \mathcal{H}$. 

The probability of $\left\lvert \Psi \right\rangle$ being described by a given state $\left\lvert \Psi_\alpha \right\rangle$ can be described by a probability map,

\begin{align*}
\pr : \left\lvert \Psi_\alpha \right\rangle & \to \text{pr} \left( \left\lvert \Psi_\alpha \right\rangle \right) \in \left[ 0, 1 \right] \\
\sum_{\alpha} \pr \left( \left\lvert \Psi_\alpha \right\rangle \right) & = 1
\end{align*}
\end{itemize}
\end{frame}

\begin{frame}{Density Operators}
\begin{itemize}
\item In general, the information contained in the possible state(s) of a quantum system are packed into what is called its \textbf{density operator} $\widehat{\rho}$,

\begin{block}{Density operators}
$$\widehat{\rho} : = \sum_{\alpha} \pr \left( \left\lvert \Psi_\alpha \right\rangle \right) \left\lvert \Psi_\alpha \right\rangle \left\langle \Psi_\alpha \right\rvert$$
\end{block}

For a pure state $\left\lvert \Psi \right\rangle$, the density operator is simply $\left\lvert \Psi \right\rangle \left\langle \Psi \right\rvert$.

\item Density operators are idempotent only for normalized pure states,

\begin{align*}
\widehat{\rho}^2 & = \left\lvert \Psi \right\rangle \left\langle \Psi \right\rvert \left. \Psi \right\rangle \left\langle \Psi \right\rvert \\
& = \left\lvert \Psi \right\rangle \left\langle \Psi \right\rvert \\
& = \widehat{\rho}
\end{align*}
\end{itemize}
\end{frame}

\begin{frame}{}
For mixed states,

\begin{align*}
\widehat{\rho}^2 & = \left[ \sum_{\alpha} \pr \left( \left\lvert \Psi_\alpha \right\rangle \right) \left\lvert \Psi_\alpha \right\rangle \left\langle \Psi_\alpha \right\rvert \right] \left[ \sum_{\beta} \pr \left( \left\lvert \Psi_\beta \right\rangle \right) \left\lvert \Psi_\beta \right\rangle \left\langle \Psi_\beta \right\rvert \right] \\
& = \sum_\alpha \sum_\beta \pr \left( \left\lvert \Psi_\alpha \right\rangle \right) \pr \left( \left\lvert \Psi_\beta \right\rangle \right) \left\lvert \Psi_\alpha \right\rangle \left\langle \Psi_\alpha \right\rvert \left. \Psi_\beta \right\rangle \left\langle \Psi_\beta \right\rvert
\end{align*}

Consider the following two cases:

\begin{enumerate}
\item $\left\langle \Psi_\alpha \right\rvert \left. \Psi_\beta \right\rangle = \delta_{\alpha \beta}$. Then, 

$$\widehat{\rho}^2 = \sum_\alpha \left[ \pr \left( \left\lvert \Psi_\alpha \right\rangle \right) \right]^2 \left\lvert \Psi_\alpha \right\rangle \left\langle \Psi_\alpha \right\rvert$$

Since by definition there are at least 2 states possibly describing a mixed state, for each state, $\pr \left( \left\lvert \Psi_\alpha \right\rangle \right) < 1$. Therefore, we can never have $\left[ \pr \left( \left\lvert \Psi_\alpha \right\rangle \right) \right]^2 = \pr \left( \left\lvert \Psi_\alpha \right\rangle \right)$. As a consequence, $\widehat{\rho}^2 \neq \widehat{\rho}$.
\end{enumerate}
\end{frame}

\begin{frame}{}
\begin{enumerate}
\setcounter{enumi}{1}
\item $\left\langle \Psi_\alpha \right\rvert \left. \Psi_\beta \right\rangle \neq \delta_{\alpha \beta}$. Then, there are non-zero entries corresponding to the indices $\alpha \neq \beta$, which is not true for $\widehat{\rho}$ as,

\begin{align*}
\widehat{\rho} & = \sum_{\alpha} \pr \left( \left\lvert \Psi_\alpha \right\rangle \right) \left\lvert \Psi_\alpha \right\rangle \left\langle \Psi_\alpha \right\rvert \\
& = \sum_\alpha \sum_\beta \pr \left( \left\lvert \Psi_\alpha \right\rangle \right) \pr \left( \left\lvert \Psi_\beta \right\rangle \right) \left\lvert \Psi_\alpha \right\rangle \delta_{\alpha \beta} \left\langle \Psi_\beta \right\rvert
\end{align*}

For $\alpha \neq \beta$, the component of $\widehat{\rho}$ corresponding to $\left\lvert \Psi_\alpha \right\rangle \left\langle \Psi_\beta \right\rvert$ is $0$. Therefore, we necessarily have $\widehat{\rho}^2 \neq \widehat{\rho}$.
\end{enumerate}

\begin{block}{Idempotence of density operators}
A density operator $\widehat{\rho}$ is \textbf{idempotent} i.e. obeys $\widehat{\rho}^2 = \widehat{\rho}$ iff the underlying state is pure and normalized.
\end{block}

\begin{itemize}
\item For future use, we define the \textbf{trace} of a linear operator,

\begin{block}{Trace of linear operators}
$$\tr \left( \widehat{A} \right) : = \sum_k \left\langle a_k \left\lvert \widehat{A} \right\rvert a_k \right\rangle = \sum_k a_k$$
\end{block}
\end{itemize}
\end{frame}

\begin{frame}{Properties of Trace}
\begin{itemize}
\item Firstly, trace is a \emph{linear} operation as,

\begin{align*}
\tr \left( c \widehat{A} \right) & : = \sum_k \left\langle a_k \left\lvert c \widehat{A} \right\rvert a_k \right\rangle \\
& = c \sum_k \left\langle a_k \left\lvert \widehat{A} \right\rvert a_k \right\rangle \\
& = c \: \tr \left( \widehat{A} \right) \\
\tr \left( \widehat{A} + \widehat{B} \right) : & = \sum_k \left\langle a_k \left\lvert \left( \widehat{A} + \widehat{B} \right) \right\rvert a_k \right\rangle \\
& = \sum_k \left[ \left\langle a_k \left\lvert \widehat{A} \right\rvert a_k \right\rangle + \left\langle a_k \left\lvert \widehat{B} \right\rvert a_k \right\rangle \right] \\
& = \sum_k \left\langle a_k \left\lvert \widehat{A} \right\rvert a_k \right\rangle + \sum_k \left\langle a_k \left\lvert \widehat{B} \right\rvert a_k \right\rangle \\
& = \tr \left( \widehat{A} \right) + \tr \left( \widehat{B} \right)
\end{align*}
\end{itemize}
\end{frame}

\begin{frame}{}
\begin{itemize}
\item Secondly, trace is a \emph{symmetric} operation,

$$\tr \left( \widehat{A} \widehat{B} \right) : = \sum_m \left\langle a_m \left\lvert \widehat{A} \widehat{B} \right\rvert a_m \right\rangle$$

\begin{align*}
& = \sum_m \left\langle a_m \right\rvert \left( \sum_i \sum_j \left\langle a_i \left\lvert \widehat{A} \right\vert a_j \right\rangle \left\lvert a_i \right\rangle \left\langle a_j \right\rvert \right) \left( \sum_k \sum_l \left\langle a_k \left\lvert \widehat{B} \right\vert a_l \right\rangle \left\lvert a_k \right\rangle \left\langle a_l \right\rvert \right) \left\lvert a_m \right\rangle \\
& = \sum_m \sum_i \sum_j \sum_k \sum_l \left\langle a_i \left\lvert \widehat{A} \right\vert a_j \right\rangle \left\langle a_k \left\lvert \widehat{B} \right\vert a_l \right\rangle \left\langle a_m \right\rvert \left. a_i \right\rangle \left\langle a_j \right\rvert \left. a_k \right\rangle \left\langle a_l \right\rvert \left. a_m \right\rangle \\
& = \sum_m \cancel{\sum_i} \sum_j \cancel{\sum_k} \cancel{\sum_l} \left\langle a_{\cancel{i}} \left\lvert \widehat{A} \right\vert a_j \right\rangle \left\langle a_{\cancel{k}} \left\lvert \widehat{B} \right\vert a_{\cancel{l}} \right\rangle \delta_{m \cancel{i}} \delta_{j \cancel{k}} \delta_{\cancel{l} m} \\
& = \sum_m \sum_j \left\langle a_{m} \left\lvert \widehat{A} \right\vert a_j \right\rangle \left\langle a_{j} \left\lvert \widehat{B} \right\vert a_{m} \right\rangle \\
& = \sum_m \sum_j \left\langle a_{j} \left\lvert \widehat{B} \right\vert a_{m} \right\rangle \left\langle a_{m} \left\lvert \widehat{A} \right\vert a_j \right\rangle \\
& = \sum_m \sum_j \left\langle a_{m} \left\lvert \widehat{B} \right\vert a_{j} \right\rangle \left\langle a_{j} \left\lvert \widehat{A} \right\vert a_{m} \right\rangle \\
& = \tr \left( \widehat{B} \widehat{A} \right)
\end{align*}
\end{itemize}
\end{frame}

\begin{frame}{}
\begin{itemize}
\item To summarize,

\begin{block}{Properties of trace}
\begin{enumerate}
\item Linearity

\begin{align*}
\tr \left( c \widehat{A} \right) & = c \tr \left( \widehat{A} \right) \\
\tr \left( \widehat{A} + \widehat{B} \right) & = \tr \left( \widehat{A} \right) + \tr \left( \widehat{B} \right)
\end{align*}

\item Symmetry

$$\tr \left( \widehat{A} \widehat{B} \right) = \tr \left( \widehat{B} \widehat{A} \right)$$
\end{enumerate}
\end{block}

\item As a corollary,

\begin{align*}
\tr \left( \left( \widehat{A}_1 \dots A_{M-1} \right) \widehat{A}_M \right) & = \tr \left( \widehat{A}_M \left( \widehat{A}_1 \dots \widehat{A}_{M-1} \right) \right) = \dots
\end{align*}

\begin{block}{Cyclicity}
$$\tr \left( \widehat{A}_1 \dots \widehat{A}_M \right) = \tr \left( \widehat{A}_M \widehat{A}_1 \dots \widehat{A}_{M-1} \right) = \tr \left( \widehat{A}_{M-1} \widehat{A}_M \widehat{A}_1 \dots \widehat{A}_{M-2} \right) = \dots$$
\end{block}
\end{itemize}
\end{frame}

\begin{frame}{Expectation Values Using Density Operators}
\begin{itemize}
\item For a pure quantum state $\left\lvert \Psi \right\rangle$,

\begin{align*}
E \left( \widehat{A} \right) = \left\langle \Psi \left\lvert \widehat{A} \right\rvert \Psi \right\rangle & = \left( \sum_k \left\langle \Psi \right. \left\lvert a_k \right\rangle \left\langle a_k \right\rvert \right) \widehat{A} \left( \sum_l \left\lvert a_l \right\rangle \left\langle a_l \right\rvert \left. \Psi \right\rangle \right) \\
& = \sum_k \sum_l \left\langle \Psi \right. \left\lvert a_k \right\rangle \left\langle a_l \right. \left\lvert \Psi \right\rangle \left\langle a_k \left\lvert \widehat{A} \right\rvert a_l \right\rangle \\
& = \sum_k \sum_l \left\langle \Psi \right. \left\lvert a_k \right\rangle \left\langle a_l \right. \left\lvert \Psi \right\rangle \left\langle a_k \left\lvert a_k \right\rvert a_k \right\rangle \\
& = \sum_k \sum_l \left\langle \Psi \right. \left\lvert a_k \right\rangle \left\langle a_l \right. \left\lvert \Psi \right\rangle \left\langle a_k \right\rvert \left. a_l \right\rangle a_k \\
& = \tr \left( \sum_k \sum_l \left\langle \Psi \right. \left\lvert a_k \right\rangle \left\langle a_l \right. \left\lvert \Psi \right\rangle \left\lvert a_l \right\rangle \left\langle a_k \right\rvert a_k \right) \\
& = \tr \left[ \left( \sum_k \left\langle \Psi \right. \left\lvert a_k \right\rangle \left\langle a_k \right\rvert \right) \left( \sum_l \left\lvert a_l \right\rangle \left\langle a_l \right\rvert \left. \Psi \right\rangle \right) \widehat{A} \right] \\
& = \tr \left( \left\lvert \Psi \right\rangle \left\langle \Psi \right\rvert \widehat{A} \right)
\end{align*}
\end{itemize}
\end{frame}

\begin{frame}{}
\begin{itemize}
\item We define the expectation value of a linear operator $\widehat{A}$ for mixed states, 

$$E \left( \widehat{A} \right) := \sum_{\alpha} \pr \left( \left\lvert \Psi_\alpha \right\rangle \right) E_\alpha \left( \widehat{A} \right)$$

where $E_\alpha \left( \widehat{A} \right)$ is denote expectation values for pure states $\left\lvert \Psi \right\rangle = \left\lvert \Psi_\alpha \right\rangle$.

\begin{align*}
E \left( \widehat{A} \right) = \sum_{\alpha} \pr \left( \left\lvert \Psi_\alpha \right\rangle \right) \tr \left( \left\lvert \Psi \right\rangle \left\langle \Psi \right\rangle \widehat{A} \right) & = \sum_{\alpha} \tr \left( \pr \left( \left\lvert \Psi_\alpha \right\rangle \right) \left\lvert \Psi \right\rangle \left\langle \Psi \right\rangle \widehat{A} \right) \\
= \tr \left( \sum_{\alpha} \pr \left( \left\lvert \Psi_\alpha \right\rangle \right) \left\lvert \Psi \right\rangle \left\langle \Psi \right\rangle \widehat{A} \right) & = \tr \left( \widehat{\rho} \widehat{A} \right)
\end{align*}

\begin{block}{Expectation values, with density operators, with Born rule}
$$E \left( \widehat{A} \right) = \tr \left( \widehat{\rho} \widehat{A} \right)$$
\end{block}

\item This leads to the following corollary,

\begin{block}{Born rule, with density operators}
$$\pr \left( \left\lvert a_k \right\rangle \right) = \tr \left( \widehat{\rho} \left\lvert a_k \right\rangle \left\langle a_k \right\rvert \right)$$
\end{block}
\end{itemize}
\end{frame}

\begin{frame}{Purity}
\begin{itemize}
\item Notice that for a normalized pure state,

\begin{align*}
\tr \left( \widehat{\rho}^2 \right) & = \tr \left( \widehat{\rho} \right) \\
& = \tr \left( \widehat{\rho} \widehat{I} \right) \\
& = E \left( \widehat{I} \right) \\
& = \left\langle \Psi \left\lvert \widehat{I} \right\rvert \Psi \right\rangle \\
& = \left\langle \Psi \right\rvert \left. \Psi \right\rangle \\
& = 1
\end{align*}

\item On the other hand, for mixed states, $\widehat{\rho}^2 \neq \widehat{\rho}$, making the above relation inapplicable. We therefore define a quantity which by construction is equal to $1$ iff the concerned quantum state is pure,

\begin{block}{Purity of density operators}
The purity of a density operator $\widehat{\rho}$ is defined as the quantity $\tr \left( \widehat{\rho}^2 \right)$.
\end{block}
\end{itemize}
\end{frame}

\subsection{Summary}
\begin{frame}{Summary}
It is useful to summarize important relationships so far, in one place:

\begin{block}{Important relationships}
\begin{enumerate}
\item Born rule (B)

$$\pr_\alpha \left( \left\lvert a_k \right\rangle \right) = \left\langle \Psi_\alpha \right. \left\lvert a_k \right\rangle \left\langle a_k \right. \left\lvert \Psi_\alpha \right\rangle = \tr \left( \left\lvert \Psi_\alpha \right\rangle \left\langle \Psi_\alpha \right\rvert \left. a_k \right\rangle \left\langle a_k \right\rvert \right)$$

\item Expectation values, without Born rule (E)

\begin{align*}
E_\alpha \left( \widehat{A} \right) & = \sum_{k} \pr_\alpha \left( \left\lvert a_k \right\rangle \right) \left\langle a_k \left\lvert \widehat{A} \right\rvert a_k \right\rangle = \tr \left( \sum_{k} \pr_\alpha \left( \left\lvert a_k \right\rangle \right) \left\lvert a_k \right\rangle \left\langle a_k \right\rvert \widehat{A} \right) \\
E \left( \widehat{A} \right) & = \sum_\alpha \pr \left( \left\lvert \Psi_\alpha \right\rangle \right) E_\alpha \left( \widehat{A} \right) \\
& = \tr \left[ \sum_\alpha \sum_k \pr \left( \left\lvert \Psi_\alpha \right\rangle \right) \pr_\alpha \left( \left\lvert a_k \right\rangle \right) \left\lvert a_k \right\rangle \left\langle a_k \right\rvert \widehat{A} \right]
\end{align*}
\end{enumerate}
\end{block}
\end{frame}

\begin{frame}{}
\begin{block}{}
\begin{enumerate}
\setcounter{enumi}{2}
\item Density operators (D)

$$\widehat{\rho} : = \sum_{\alpha} \pr \left( \left\lvert \Psi_\alpha \right\rangle \right) \left\lvert \Psi_\alpha \right\rangle \left\langle \Psi_\alpha \right\rvert$$

\item Born rule, with expectation values (BE)

$$\pr \left( \left\lvert a_k \right\rangle \right) = E \left( \left\lvert a_k \right\rangle \left\langle a_k \right\rvert \right)$$

\item Born rule, with density operators (BD)

$$\pr \left( \left\lvert a_k \right\rangle \right) = \tr \left( \widehat{\rho} \left\lvert a_k \right\rangle \left\langle a_k \right\rvert \right)$$

\item Expectation values, with Born rule (EB)

$$E_\alpha \left( \widehat{A} \right) = \left\langle \Psi_\alpha \left\lvert \widehat{A} \right\rvert \Psi_\alpha \right\rangle = \tr \left( \left\lvert \Psi_\alpha \right\rangle \left\langle \Psi_\alpha \right\rvert \widehat{A} \right)$$

\item Expectation values, with density operators, with Born rule (EDB)

$$E \left( \widehat{A} \right) = \tr \left( \widehat{\rho} \widehat{A} \right)$$

\item Purity of density operators (P)

$$\left[ \tr \left( \widehat{\rho}^2 \right) = 1 \right] \Longleftrightarrow \widehat{\rho} \text{ describes a pure state}$$
\end{enumerate}
\end{block}
\end{frame}

\begin{frame}{}
\begin{itemize}
\item Thus, we start seeing connections between:

\begin{enumerate}
\item Born rule (B)
\item Expectation values (E)
\item Density operators (D)
\end{enumerate}

\item It is easy to check that we are yet to state/establish the following relationships: ED, DB, DE, BED, BDE, EBD, DBE, DEB. 

\item It turns out that BD (and equivalently BE) implies B (by considering pure states). Similarly, EDB implies EB.

\item Furthermore, since E and D are independent ideas when built from first principles, it can be shown that ED, DE, DBE, DEB and EBD are not possible. 

This is because relating E and D using usual algebraic tools requires B as a \emph{priori}; however, this is subtle and will be explored in detail later into the slides.

As a consequence, BED and BDE are not possible at the moment either.

\item To summarize, the independent relationships found so far are: \textbf{E}, \textbf{D}, \textbf{BE}, \textbf{BD} and \textbf{EDB}.

On the other hand, by the above analysis, there \emph{are} no other relationships that can be found using the tools constructed so far!
\end{itemize}
\end{frame}

\section{Transition Between States}

\subsection{Normalization}
\begin{frame}{Normalization}
\begin{itemize}
\item Recall that $\pr \left( \left\lvert a_k \right\rangle \right)$ represents the probability that the measurement of an appropriate observable $\widehat{A}$ makes a quantum state represented by a density operator $\widehat{\rho}$ collapse to the eigenstate $\left\lvert a_k \right\rangle$.

\item Consider a complex number $c \in \mathbb{C}$. We ask, what is the probability that the state collapses to $c \left\lvert a_k \right\rangle$ after measuring $\widehat{A}$?

\item It turns out that the relationships so far \emph{fail} if we use them as-is (in particular, Born rule with density operators i.e. BD):

\begin{align*}
\pr \left( c \left\lvert a_k \right\rangle \right) & = \tr \left( \widehat{\rho} c \left\lvert a_k \right\rangle \left\langle a_k \right\rvert c^* \right) \\
& = \tr \left( c^* c \widehat{\rho} \left\lvert a_k \right\rangle \left\langle a_k \right\rvert \right) \\
& = c^* c \: \tr \left( \widehat{\rho} \left\lvert a_k \right\rangle \left\langle a_k \right\rvert \right) \\
& = c^* c \: \pr \left( \left\lvert a_k \right\rangle \right)
\end{align*}

But as $c$ is arbitrary, the above probability is not normalized, which is clearly not consistent.

\item The key conflict here is that traces are linear, whereas probabilities over states should not be. This is further seen in the fact that scaling a state does not make it linearly independent to the original state, therefore it is really the same physical state.
\end{itemize}
\end{frame}

\begin{frame}{}
\begin{itemize}
\item To fix this problem, we must rewrite all relationships so far by performing \textbf{normalization}:

\begin{align*}
\left\lvert \Psi \right\rangle & \to \frac{1}{\left[ \left\langle \Psi \left\lvert \right. \Psi \right\rangle \right]^{1/2}} \left\lvert \Psi \right\rangle \\
& = \frac{1}{\left[ \tr \left( \left\lvert \Psi \right\rangle \left\langle \Psi \right\rvert \right) \right]^{1/2}} \left\lvert \Psi \right\rangle
\end{align*}

\item We could have restricted states to be unit vectors in $\mathbb{H}$ to begin with, but the reason for adopting the above approach is twofold:

\begin{enumerate}
\item It is satisfying to have that the norm of a state does not \emph{matter}, thereby generalizing the domain of states to all of $\mathbb{H}$. This also encodes the physical equivalence of linearly dependent states.

\item By considering how probabilities behave over linear combinations of states, we can see a hint of transition amplitudes between entire density states and not just from density states to eigenstates.
\end{enumerate}

\item Now, let us rewrite the important relationships mentioned earlier, with explicit normalization where required.
\end{itemize}
\end{frame}

\begin{frame}{}
\begin{block}{Important normalized relationships}
\begin{enumerate}
\item Density operators, normalized (DN)

$$\widehat{\rho} : = \sum_\alpha \frac{\pr \left( \left\lvert \Psi_\alpha \right\rangle \right)}{\tr \left( \left\lvert \Psi_\alpha \right\rangle \left\langle \Psi_\alpha \right\rvert \right)} \left\lvert \Psi_\alpha \right\rangle \left\langle \Psi_\alpha \right\rvert$$

\item Born rule, with expectation values, normalized (BEN)

$$\pr \left( c \left\lvert a_k \right\rangle \right) = \frac{E \left( c \left\lvert a_k \right\rangle \left\langle a_k \right\rvert c^* \right)}{\tr \left( c \left\lvert a_k \right\rangle \left\langle a_k \right\rvert c^* \right)}$$

\item Born rule, with density operators, normalized (BDN)

$$\pr \left( c \left\lvert a_k \right\rangle \right) = \frac{\tr \left( \widehat{\rho} c \left\lvert a_k \right\rangle \left\langle a_k \right\rvert c^* \right)}{\tr \left( c \left\lvert a_k \right\rangle \left\langle a_k \right\rvert c^* \right)}$$
\end{enumerate}
\end{block}

\begin{itemize}
\item It can be shown that EDB follows from DN, E, BEN and BDN. Furthermore, normalizing E and EDB is not required.

\item We can now handle calculations concerning measurement of a normalized mixed state to not only any scaled version of some eigenstate, but also the transition of these mixed states to entire mixed states!
\end{itemize}
\end{frame}

\subsection{Transition to Pure States}
\begin{frame}{Transition to Pure States}
\begin{itemize}
\item We will now interpret the normalized relationships above backwards, start with the right hand side of BDN, use its linear properties for sums of projectors and end up with the probability of finding a superposition of states generalizing the left hand side,

\begin{align*}
\pr_{\text{sup}} \left( \left\lvert \Phi \right\rangle \right) & : = \pr \left( \sum_k \left\lvert a_k \right\rangle \left\langle a_k \right\rvert \left. \Phi \right\rangle \right) \\
& = \frac{\tr \left[ \widehat{\rho} \left( \sum_k \left\lvert a_k \right\rangle \left\langle a_k \right\rvert \left. \Phi \right\rangle \right) \left( \sum_l \left\langle \Phi \right\rvert \left. a_l \right\rangle \left\langle a_l \right\rvert \right) \right]}{\tr \left[ \left( \sum_k \left\lvert a_k \right\rangle \left\langle a_k \right\rvert \left. \Phi \right\rangle \right) \left( \sum_l \left\langle \Phi \right\rvert \left. a_l \right\rangle \left\langle a_l \right\rvert \right) \right]} \\
& = \frac{\tr \left( \widehat{\rho} \left\lvert \Phi \right\rangle \left\langle \Phi \right\rvert \right)}{\tr \left( \left\lvert \Phi \right\rangle \left\langle \Phi \right\rvert \right)}
\end{align*}

\item We interpret the above relationship to quantify the probability associated with the transition of a mixed state to a pure state. 
\end{itemize}
\end{frame}

\begin{frame}{}
\begin{itemize}
\item When the initial state is a pure state $\left\lvert \Psi \right\rangle$,

\begin{align*}
\pr \left( \left\lvert \Phi \right\rangle \right) & = \frac{\tr \left( \frac{1}{\tr \left( \left\lvert \Psi \right\rangle \left\langle \Psi \right\rvert \right)} \left\lvert \Psi \right\rangle \left\langle \Psi \right\rvert \left. \Phi \right\rangle \left\langle \Phi \right\rvert \right)}{\tr \left( \left\lvert \Phi \right\rangle \left\langle \Phi \right\rvert \right)} \\
& = \frac{\tr \left( \left\lvert \Psi \right\rangle \left\langle \Psi \right\rvert \left. \Phi \right\rangle \left\langle \Phi \right\rvert \right)}{\tr \left( \left\lvert \Psi \right\rangle \left\langle \Psi \right\rvert \right) \tr \left( \left\lvert \Phi \right\rangle \left\langle \Phi \right\rvert \right)} \\
& = \frac{\left\langle \Psi \right\rvert \left. \Phi \right\rangle \left\langle \Phi \right\rvert \left. \Psi \right\rangle}{\left\langle \Psi \right\rvert \left. \Psi \right\rangle \left\langle \Phi \right\rvert \left. \Phi \right\rangle}
\end{align*}

This is a generalized form of the Born rule which does not directly involve measurement. Here, $\displaystyle{\frac{\left\langle \Psi \right\rvert \left. \Phi \right\rangle}{\left\langle \Psi \right\rvert \left. \Psi \right\rangle \left\langle \Phi \right\rvert \left. \Phi \right\rangle}}$ is interpreted as the transition amplitude corresponding to initial and final states $\left\lvert \Psi \right\rangle$ and $\left\lvert \Phi \right\rangle$, respectively.

\item We now adopt the following notation for transition from an initial density to a final density (for a pure state at this stage),

\begin{block}{Transition to pure states}
$$\pr \left( \widehat{\rho}; \left\lvert \Phi \right\rangle \left\langle \Phi \right\rvert \right) : = \frac{\tr \left( \widehat{\rho} \left\lvert \Phi \right\rangle \left\langle \Phi \right\rvert \right)}{\tr \left( \left\lvert \Phi \right\rangle \left\langle \Phi \right\rvert \right)}$$
\end{block}
\end{itemize}
\end{frame}

\subsection{Transition Between Mixed States}
\begin{frame}{Transition Between Mixed States}
\begin{itemize}
\item Once again, we use the linear properties involved in expressions obtained for probability to compute the probability of transition from a mixed state with a density $\widehat{\rho}$ to another mixed state with a density $\displaystyle{\widehat{\rho^\prime} = \sum_\alpha \pr \left( \widehat{\rho^\prime}; \left\lvert \Phi_\alpha \right\rangle \right) \left\lvert \Phi_\alpha \right\rangle \left\langle \Phi_\alpha \right\rvert}$,

\begin{align*}
\pr_{\text{mixed}} \left( \left\{ \left\lvert \Phi_\alpha \right\rangle \right\}, \left\{ \pr \left( \left\lvert \Phi_\alpha \right\rangle \right) \right\} \right) : & = \pr \left( \widehat{\rho}; \sum_\alpha \pr \left( \widehat{\rho^\prime}; \left\lvert \Phi_\alpha \right\rangle \right) \left\lvert \Phi_\alpha \right\rangle \left\langle \Phi_\alpha \right\rvert \right) \\
& = \sum_\alpha \pr \left( \widehat{\rho^\prime}; \left\lvert \Phi_\alpha \right\rangle \right) \pr \left( \widehat{\rho}; \left\lvert \Phi_\alpha \right\rangle \left\langle \Phi_\alpha \right\rvert \right) \\
& = \sum_\alpha \pr \left( \widehat{\rho^\prime}; \left\lvert \Phi_\alpha \right\rangle \right) \frac{\tr \left( \widehat{\rho} \left\lvert \Phi_\alpha \right\rangle \left\langle \Phi_\alpha \right\rvert \right)}{\tr \left( \left\lvert \Phi_\alpha \right\rangle \left\langle \Phi_\alpha \right\rvert \right)} \\
& = \tr \left[ \sum_\alpha \frac{\pr \left( \widehat{\rho^\prime}; \left\lvert \Phi_\alpha \right\rangle \right)}{\tr \left( \left\lvert \Phi_\alpha \right\rangle \left\langle \Phi_\alpha \right\rvert \right)} \widehat{\rho} \left\lvert \Phi_\alpha \right\rangle \left\langle \Phi_\alpha \right\rvert \right] \\
& = \tr \left[ \widehat{\rho} \sum_\alpha \frac{\pr \left( \widehat{\rho^\prime}; \left\lvert \Phi_\alpha \right\rangle \right)}{\tr \left( \left\lvert \Phi_\alpha \right\rangle \left\langle \Phi_\alpha \right\rvert \right)} \left\lvert \Phi_\alpha \right\rangle \left\langle \Phi_\alpha \right\rvert \right]
\end{align*}
\end{itemize}
\end{frame}

\begin{frame}{}
$$ = \tr \left( \widehat{\rho} \: \widehat{\rho^\prime} \right)$$

\begin{itemize}
\item Using notation similar to before, we write the probability of transition from a mixed state with density $\widehat{\rho}$ to that with density $\widehat{\rho^\prime}$ as,

\begin{block}{Transition between mixed states}
$$\pr \left( \widehat{\rho}; \widehat{\rho^\prime} \right) = \tr \left( \widehat{\rho} \widehat{\rho^\prime} \right)$$
\end{block}

\item An immediate and interesting consequence of the above is symmetry,

\begin{block}{Transition symmetry}
$$\pr \left( \widehat{\rho}; \widehat{\rho^\prime} \right) = \pr \left( \widehat{\rho^\prime}; \widehat{\rho} \right)$$
\end{block}

Therefore, if states are, for example, parameterized by time, so are density operators, and the above fact translates to a form of time symmetry in quantum mechanics.
\end{itemize}
\end{frame}

\section{Reinterpreting the Born Rule}
\begin{frame}{Reinterpreting the Born Rule}
\begin{itemize}
\item Now is a good time to pause and question what is going on. 

\item We went all the way from the probability of transitions from mixed states to eigenstates — to that of transitions from mixed states to superpositions of eigenstates comprising pure states — to that of transitions from mixed states to mixed states!

\item We performed the above by interpreting that since the probabilities mentioned were in the form of traces acting on projectors and traces are linear, we can add projectors corresponding to a superposition of the appropriate eigenstates.

Which begs the question, WHY?

\item One way to answer this question is to say we were handwaving and it turned out the final notion made sense. For example, transition amplitudes in the form of pure final states, seen before, are crucial to traditional quantum mechanics and 'make sense' just as much as the usual constructions do.

\item However, such an answer may be justifiably unsatisfactory to many people. Therefore, similar to much of the climb to abstraction in the history of mathematical physics, we will now go in reverse gear!
\end{itemize}
\end{frame}

\subsection{Principles of Measurement}
\begin{frame}{Principles of Measurement}
\begin{itemize}
\item Since going from transition of states to eigenstates, to transition of states to entire ensembles of superimposed states may seem like cheating, we can axiomatize the latter (transition between mixed states) and recover the former (Born rule). This is fairly straightforward:

$$\pr \left( \widehat{\rho}; \left\lvert a_k \right\rangle \left\langle a_k \right\rvert \right) = \tr \left( \widehat{\rho} \left\lvert a_k \right\rangle \left\langle a_k \right\rvert \right)$$

This is identical to the previously-explored condition of BD (Born rule, with density operators).

\item It turns out that an alternative but more constructive way to derive the Born rule is using new principles of measurement: wavefunction collapse and the measurement invariance of density operators.
\end{itemize}
\end{frame}

\begin{frame}{}
\begin{block}{Principles of Measurement}
\begin{enumerate}
\item \textbf{Wavefunction collapse}: After measurement of an observable $\widehat{A}$, a mixed state described by states $\left\{ \left\lvert \Psi_\alpha \right\rangle \right\}$ and a density $\widehat{\rho}$ transitions in the manner,

\begin{align*}
\widehat{\rho} \to \widehat{\sigma} : & = \sum_\alpha \sum_k \pr \left( \left\lvert \Psi_\alpha \right\rangle \right) \pr \left( \left\lvert \Psi_\alpha \right\rangle \left\langle \Psi_\alpha \right\rvert; \left\lvert a_k \right\rangle \left\langle a_k \right\rvert \right) \left\lvert a_k \right\rangle \left\langle a_k \right\rvert \\
\pr \left( \left\lvert \Psi_\alpha \right\rangle \right) & = \pr \left( \widehat{\sigma}; \left\lvert \Psi_\alpha \right\rangle \right)
\end{align*}

\item \textbf{Measurement invariance of density operators}: A measurement does not change the density operator! I.e.,

$$\widehat{\sigma} = \widehat{\rho}$$
\end{enumerate}
\end{block}

Let us briefly look at wavefunction collapse before recovering the Born rule using it and the measurement invariance of density operators!
\end{frame}

\begin{frame}{Wavefunction Collapse}
\begin{itemize}
\item In the Copenhagen interpretation of quantum mechanics, the measurement of an observable $\widehat{A}$ makes a quantum state prepared in some pure state $\left\lvert \Psi \right\rangle$ probabilistically collapse to an eigenstate $\left\lvert a_k \right\rangle$.

\item Immediately after measurement, the new state resembles a mixed state (but behaves like a pure state, as we'll see later) with the eigenstates as its constituent states,

$$\widehat{\sigma} = \sum_k \pr \left( \left\lvert \Psi \right\rangle \left\langle \Psi \right\rvert; \left\lvert a_k \right\rangle \right) \left\lvert a_k \right\rangle \left\langle a_k \right\rvert$$

\item For a mixed state in a density $\widehat{\rho}$, the new state is a weighed sum of eigenstates with the weights being conditional probabilities corresponding to the states packed into the initial density,

\begin{align*}
\widehat{\sigma} & = \sum_\alpha \sum_k \pr \left( \widehat{\sigma}; \left\lvert \Psi_\alpha \right\rangle \left\langle \Psi_\alpha \right\rvert \right) \pr \left( \left\lvert \Psi_\alpha \right\rangle \left\langle \Psi_\alpha \right\rvert; \left\lvert a_k \right\rangle \left\langle a_k \right\rvert \right) \left\lvert a_k \right\rangle \left\langle a_k \right\rvert \\
& = \sum_\alpha \sum_k \sum_l \pr \left( \widehat{\sigma}; \left\lvert \Psi_\alpha \right\rangle \left\langle \Psi_\alpha \right\rvert \right) \pr \left( \left\lvert \Psi_\alpha \right\rangle \left\langle \Psi_\alpha \right\rvert; \left\lvert a_k \right\rangle \left\langle a_k \right\rvert \right) \delta_{k l} \left\lvert a_k \right\rangle \left\langle a_l \right\rvert
\end{align*}
\end{itemize}
\end{frame}

\subsection{Rederiving the Born Rule}
\begin{frame}{Rederiving the Born Rule}
\begin{itemize}
\item Consider the density operator for a mixed state before measurement,

\begin{align*}
\widehat{\rho} & = \sum_\alpha \pr \left( \widehat{\rho}; \left\lvert \Psi_\alpha \right\rangle \left\langle \Psi_\alpha \right\rvert \right) \left\lvert \Psi_\alpha \right\rangle \left\langle \Psi_\alpha \right\rvert \\
& = \sum_\alpha \pr \left( \widehat{\rho}; \left\lvert \Psi_\alpha \right\rangle \left\langle \Psi_\alpha \right\rvert \right) \left( \sum_k \left\lvert a_k \right\rangle \left\langle a_k \right\rvert \left. \Psi \right\rangle \right) \left( \sum_l \left\langle \Psi \right\rvert \left. a_l \right\rangle \left\langle a_l \right\rvert \right) \\
& = \sum_\alpha \sum_k \sum_l \pr \left( \widehat{\rho}; \left\lvert \Psi_\alpha \right\rangle \left\langle \Psi_\alpha \right\rvert \right) \left\langle a_k \right\rvert \left. \Psi \right\rangle \left\langle \Psi \right\rvert \left. a_l \right\rangle \left\lvert a_k \right\rangle \left\langle a_l \right\rvert
\end{align*}

\item Recall from the principles of measurement that the density operator is unchanged after measurement i.e. $\widehat{\sigma} = \widehat{\rho}$,

\begin{align*}
& \sum_\alpha \sum_k \sum_l \pr \left( \widehat{\sigma}; \left\lvert \Psi_\alpha \right\rangle \left\langle \Psi_\alpha \right\rvert \right) \pr \left( \left\lvert \Psi_\alpha \right\rangle \left\langle \Psi_\alpha \right\rvert; \left\lvert a_k \right\rangle \left\langle a_k \right\rvert \right) \delta_{k l} \left\lvert a_k \right\rangle \left\langle a_l \right\rvert \\
= & \sum_\alpha \sum_k \sum_l \pr \left( \widehat{\rho}; \left\lvert \Psi_\alpha \right\rangle \left\langle \Psi_\alpha \right\rvert \right) \left\langle a_k \right\rvert \left. \Psi \right\rangle \left\langle \Psi \right\rvert \left. a_l \right\rangle \left\lvert a_k \right\rangle \left\langle a_l \right\rvert
\end{align*}
\end{itemize}
\end{frame}

\begin{frame}{}
Since the above must be true for pure and mixed states alike,

\begin{align*}
\sum_k \sum_l \pr \left( \left\lvert \Psi_\alpha \right\rangle \left\langle \Psi_\alpha \right\rvert; \left\lvert a_k \right\rangle \left\langle a_k \right\rvert \right) \delta_{k l} \left\lvert a_k \right\rangle \left\langle a_l \right\rvert & = \sum_k \sum_l \left\langle a_k \right\rvert \left. \Psi \right\rangle \left\langle \Psi \right\rvert \left. a_l \right\rangle \left\lvert a_k \right\rangle \left\langle a_l \right\rvert \\
\pr \left( \left\lvert \Psi_\alpha \right\rangle \left\langle \Psi_\alpha \right\rvert; \left\lvert a_k \right\rangle \left\langle a_k \right\rvert \right) \delta_{k l} & = \left\langle a_k \right\rvert \left. \Psi \right\rangle \left\langle \Psi \right\rvert \left. a_l \right\rangle
\end{align*}

Setting $k = l$, we recover the Born rule,

$$\pr \left( \left\lvert \Psi_\alpha \right\rangle \left\langle \Psi_\alpha \right\rvert; \left\lvert a_k \right\rangle \left\langle a_k \right\rvert \right) = \left\langle \Psi \right\rvert \left. a_k \right\rangle \left\langle a_k \right\rvert \left. \Psi \right\rangle$$
\end{frame}

\subsection{Determinism}
\begin{frame}{Determinism}
\begin{itemize}
\item We have seen so far how the Born rule can be derived from either transition between mixed states, or new principles of quantum measurement.

\item For the purpose of describing the Born rule alone, transition between states is a more efficient assumption than principles of measurement, but on the other hand, the measurement principles provide more of a 'mechanism' underlying measurement than transition amplitudes.

\item To begin seeing this mechanism, consider both transition between states and measurement principles. If we start with a density $\widehat{\rho}$ and apply a measurement to get a density $\widehat{\sigma}$, we must have $\widehat{\rho} = \widehat{\sigma}$. Therefore,

\begin{align*}
\pr \left( \widehat{\rho}; \widehat{\sigma} \right) & = \pr \left( \widehat{\rho}; \widehat{\rho} \right) \\
& = \tr \left( \widehat{\rho}^2 \right) 
\end{align*}
\end{itemize}
\end{frame}

\begin{frame}{}
\begin{block}{Transition during measurement}
The probability associated with the [invariant] transition of a density operator induced by a measurement is equal to the purity of the density operator.

$$\pr \left( \widehat{\rho}; \widehat{\sigma} \right) = \tr \left( \widehat{\rho}^2 \right)$$

\textbf{Corollary}: The transition of a state under measurement is \emph{deterministic} iff the state is pure,

$$\pr \left( \left\lvert \Psi \right\rangle \left\langle \Psi \right\rvert; \widehat{\sigma} \right) = 1$$

where,

$$\widehat{\sigma} = \sum_k \pr \left( \left\lvert \Psi \right\rangle \left\langle \Psi \right\rvert; \left\lvert a_k \right\rangle \right) \left\lvert a_k \right\rangle \left\langle a_k \right\rvert$$
\end{block}

\begin{itemize}
\item A natural interpretation of the above statement along with the underlying mechanism of measurement is the Many Worlds Interpretation of quantum mechanics.
\end{itemize}
\end{frame}

\subsection{Unitarity}
\begin{frame}{Unitarity}
\begin{itemize}
\item In the previous slide, we have shown that quantum measurement is, in a sense, \emph{deterministic} during the measurement of strictly pure states.

\item Does this violate quantum mechanics? The answer is 'NO'. 
Collapse to individual eigenstates still involves non-unital probabilities in the above framework.

\item The specific transition amplitude which yielded a probability of unity corresponds to that of a measurement-induced transition from an initial pure state to a final 'mixed' state of the eigenstates, weighed with the probabilities of transitioning to these eigenstates.

\item The above scenario turned out to be deterministic as all eigenstates, with their own non-unital transition probabilities, are accounted for in the final, apparently mixed state, which really turns out to be pure.

\item In fact, this phenomenon is equivalent to unitarity, at least for pure states.
\end{itemize}
\end{frame}

\section{References}
\begin{frame}{References}
\begin{enumerate}
\item \emph{Quantum Mechanics: The Theoretical Minimum}, Leonard Susskind
\item \emph{The Principles of Quantum Mechanics}, Paul Dirac
\item \emph{Mathematical Foundations of Quantum Mechanics}, John von Neumann
\item \emph{Quantum Physics: Symbolism of Atomic Measurements}, Julian Schwinger
\item \href{https://ncatlab.org/nlab/show/HomePage}{nLab}
\item \href{https://plato.stanford.edu/}{Stanford Encyclopedia of Philosophy}
\item \href{https://www.wikipedia.org/}{Wikipedia}
\end{enumerate}
\end{frame}

%% References
%\appendix
%\begin{frame}[allowframebreaks]
%        \frametitle{References}
%        \printbibliography
%\end{frame}
\end{document}